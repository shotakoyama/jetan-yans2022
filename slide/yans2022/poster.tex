\documentclass[12pt]{beamer}

% beamer posterの利用に関する設定
% A0 に合わせて 4 : 3 に (112cm x 84cm)
\usepackage[orientation=landscape,size=custom,width=112,height=84,scale=2.1]{beamerposter}

% メインフォントに関する設定
\usepackage{fontspec}
\setmainfont[Ligatures=TeX,BoldFont=Noto Sans Bold]{Noto Sans}

% CJKに対応させる
% 太字フォントを指定しないと疑似ボールドになる
\usepackage{xeCJK}
\setCJKmainfont[BoldFont=Noto Sans CJK JP Bold]{Noto Sans CJK JP}

% 追加のフォント
\newfontfamily{\dejavu}{DejaVu Sans}
\newfontfamily{\notosansmono}{Noto Sans Mono}

% tikz
\usepackage{tikz}
\usepackage{pgfplots}
\usepackage{bchart}
\usetikzlibrary{positioning}
\usetikzlibrary{arrows.meta}

% beamerのテーマ・スタイルに関する設定
\setbeamertemplate{navigation symbols}{} % 右下のナビゲーションアイコンに消えてもらう
\setbeamercolor{normal text}{fg=black,bg=white}
\setbeamercolor{structure}{fg=red}

% blockのスタイル
\usepackage[most]{tcolorbox} % mostがないとenhanced(tikzを用いる装飾表現)が使えない
\setbeamertemplate{block begin}{
	\begin{tcolorbox}[
				enhanced,
				fonttitle=\bfseries,
				colframe=gray,
				colback=gray!5!,
				colbacktitle=gray!80,
				attach boxed title to top left={yshift=-2mm, xshift=2mm},
				title=\insertblocktitle
			]
		\vskip2mm
}
\setbeamertemplate{block end}{
	\end{tcolorbox}
}

% itemize
\setlength\leftmargini{1.5em}
\setlength\leftmarginii{0.75em}
\setlength\leftmarginiii{0.75em}
\setbeamertemplate{itemize item}{\hbox{\textcolor{blue}{$\blacktriangleright$}}}
\setbeamertemplate{itemize subitem}{\hbox{\textcolor{orange}{$\blacktriangleright$}}}
\setbeamertemplate{itemize subsubitem}{\hbox{\textcolor{cyan}{$\blacktriangleright$}}}
\setbeamerfont{itemize/enumerate body}{size=\normalsize}
\setbeamerfont{itemize/enumerate subbody}{size=\normalsize}
\setbeamerfont{itemize/enumerate subsubbody}{size=\normalsize}


% href
\usepackage{hyperref}
\hypersetup{
	colorlinks=true,
	linkcolor=blue,
	filecolor=magenta,
	urlcolor=cyan,
	pdfnewwindow=true}

% その他
\usepackage{fancybox}
\usepackage{here}


\newenvironment{mybox}{%
	\begin{tcolorbox}[%
			enhanced,
			fonttitle=\bfseries,
			colframe=gray,
			colback=gray!25!,
			colbacktitle=gray!80,
			attach boxed title to top left={yshift=-2mm, xshift=2mm},
			]
		}{%
	\end{tcolorbox}
}

\newcommand{\mytitle}[1]{%
	\begin{figure}[H]
		\centering
		\begin{tikzpicture}[
				font=\sffamily,
				scale=1.1,
				transform shape]
			\node[
				rectangle,
				text width=180,
				text height=50,
				fill=blue!20,
				draw=black,
				outer sep=1mm
			] at (0, 0) {};
			\node at (0, 0) {\textbf{#1}};
		\end{tikzpicture}
	\end{figure}
}


\newcommand{\figa}{%
	\begin{figure}[H]
		\centering
		\begin{tikzpicture}[
				font=\sffamily,
				scale=1.05,
				transform shape]
			\node[
				rectangle,
				text width=190,
				text height=20,
				draw=green!20,
				fill=green!20,
				outer sep=1mm
			] (s) at (3, 5) {};
			\node[
				rectangle,
				text width=180,
				text height=10,
				draw=gray!80,
				fill=gray!80,
				outer sep=1mm
			] (sa) at (2.5, 6.5) {};
			\node[
				rectangle,
				text width=270,
				text height=20,
				draw=green!20,
				fill=green!20,
				outer sep=1mm
			] (t) at (22, 0.6) {};
			\node[
				rectangle,
				text width=180,
				text height=10,
				draw=gray!80,
				fill=gray!80,
				outer sep=1mm
			] (ta) at (19.5, -0.9) {};
			\node[
				rectangle,
				text width=270,
				text height=20,
				draw=green!20,
				fill=green!20,
				outer sep=1mm
			] (r) at (22, 5) {};
			\node[
				rectangle,
				text width=90,
				text height=10,
				draw=gray!80,
				fill=gray!80,
				outer sep=1mm
			] (ra) at (18.0, 6.5) {};
			%\node[
			%	rectangle,
			%	text width=140,
			%	text height=55,
			%	rounded corners,
			%	draw=blue,
			%	fill=blue!20,
			%	outer sep=1mm
			%] (m) at (9, 0.8) {};
			\node (i) at (2.5, 6.5) {\footnotesize \color{white!20} \textbf{モデルの入力}};
			\node (k) at (18.0, 6.5) {\footnotesize \color{white!20} \textbf{参照文}};
			\node (j) at (19.5, -0.9) {\footnotesize \color{white!20} \textbf{モデルの出力}};
			\node (x) at (3, 5) {\small きれくない};
			\node (z) at (22, 5) {\small \textcolor{gray}{\textbf{きれい}}\textcolor{violet}{\textbf{じゃ}}ない};
			\node (y) at (22, 0.6) {\small \textcolor{gray}{\textbf{きれい}}\textcolor{violet}{\textbf{では}}ない};
			%\node at (9, 1.6) {\small 誤り訂正};
			%\node at (9, 0) {\small モデル};
			\node at (27, 2.8) {\small \color{cyan} 比較・評価};
			\node at (8, 1.0) {\small \color{cyan} 区間の推定};
			%\draw[-{Triangle[width=30pt,length=16pt]}, line width=2pt] (x.south) -- (3, 0.8) -- (m.west);
			%\draw[-{Triangle[width=30pt,length=16pt]}, line width=2pt] (12, 0.6) -- (t.west);
			\node at (18.0, 2.8) {\small \color{green} o};
			\node at (21.5, 2.8) {\small \color{red} x};
			\draw[{Triangle[width=30pt,length=16pt]}-{Triangle[width=30pt,length=16pt]}, line width=2pt] (19.0, 1.6) -- (19.0, 4.0);
			\draw[{Triangle[width=30pt,length=16pt]}-{Triangle[width=30pt,length=16pt]}, line width=2pt] (22.5, 1.6) -- (22.5, 4.0);
			\draw[{Triangle[width=30pt,length=16pt]}-{Triangle[width=30pt,length=16pt]}, line width=2pt] (7, 5) -- (16.5, 5);
			\draw[{Triangle[width=30pt,length=16pt]}-{Triangle[width=30pt,length=16pt]}, line width=2pt] (7, 4.5) -- (16.5, 1.1);
		\end{tikzpicture}
	\end{figure}
}


\newcommand{\figb}{%
	\begin{figure}[H]
		\centering
		\begin{tikzpicture}[
				font=\sffamily,
				scale=1.05,
				transform shape]
			\node[
				rectangle,
				text width=190,
				text height=20,
				draw=green!20,
				fill=green!20,
				outer sep=1mm
			] (s) at (3, 0) {};
			\node[
				rectangle,
				text width=150,
				text height=10,
				draw=gray!80,
				fill=gray!80,
				outer sep=1mm
			] (sa) at (1, 1.5) {};
			\node[
				rectangle,
				text width=270,
				text height=20,
				draw=green!20,
				fill=green!20,
				outer sep=1mm
			] (t) at (20, 0) {};
			\node[
				rectangle,
				text width=190,
				text height=10,
				draw=gray!80,
				fill=gray!80,
				outer sep=1mm
			] (ta) at (17, 1.5) {};
			\node (i) at (1, 1.5) {\footnotesize \color{white!20} \textbf{学習者の文}};
			\node (j) at (17, 1.5) {\footnotesize \color{white!20} \textbf{教師の訂正文}};
			\node (x) at (3, 0) {\small \textcolor{gray}{\textbf{きれく}}ない};
			\node (y) at (20, 0) {\small \textcolor{gray}{\textbf{きれい}}\textcolor{violet}{\textbf{じゃ}}ない};
			\draw[-{Triangle[width=30pt,length=16pt]}, line width=2pt] (1.5, -1) -- (1.5, -2.5) -- (17, -2.5) -- (17, -1);
			\draw[-{Triangle[width=30pt,length=16pt]}, line width=2pt] (21, -2.8) -- (21, -1);
			\node (p) at (9.25, -3.5) {\footnotesize \color{blue} イ形容詞・ナ形容詞の混同};
			\node (q) at (21, -3.5) {\footnotesize \color{blue} 脱落};
		\end{tikzpicture}
	\end{figure}
}


\newcommand{\figc}{%
	\begin{figure}[H]
		\centering
		\begin{tikzpicture}[
				font=\sffamily,
				scale=1.0,
				transform shape]
			\node[
				rectangle,
				text width=250,
				text height=15,
				draw=green!20,
				fill=green!20,
				outer sep=1mm
			] (a) at (0, 0) {};
			\node[
				rectangle,
				text width=130,
				text height=15,
				draw=green!20,
				fill=green!20,
				outer sep=1mm
			] (b) at (11, 0) {};
			\draw[-{Triangle[width=30pt,length=16pt]}, line width=2pt] (-7, 0) -- (a.west);
			\draw[-{Triangle[width=30pt,length=16pt]}, line width=2pt] (a) -- node [auto] {\hspace*{-0.5em}\color{blue}Yes} (b);
			\draw[-{Triangle[width=30pt,length=16pt]}, line width=2pt] (b.east) -- node [auto] {\hspace*{-0.5em}\color{blue}Yes} (17, 0);
			\draw[-{Triangle[width=30pt,length=16pt]}, line width=2pt] (a.south) -- (0, -2) -- (3, -2);
			\draw[-{Triangle[width=30pt,length=16pt]}, line width=2pt] (b.south) -- (11, -2) -- (14, -2);
			\node at (-1.5, -1.7) {\color{red}No};
			\node at (9.5, -1.7)  {\color{red}No};
			\node at (-9, 0) {\small マージ};
			\node at (0, 0) {\small 隣接する編集か};
			\node at (11,0) {\small MWEか};
			\node at (19,0) {\small する};
			\node at (5,-2) {\small しない};
			\node at (16,-2) {\small しない};
		\end{tikzpicture}
	\end{figure}
}


\newcommand{\figd}{%
	\begin{figure}[H]
		\centering
		\begin{tikzpicture}[
				font=\sffamily,
				scale=1.0,
				transform shape]

			\node[
				rectangle,
				text width=100,
				text height=15,
				draw=green!20,
				fill=green!20,
				outer sep=1mm
			] (a) at (0, 0) {};

			\node[
				rectangle,
				text width=190,
				text height=50,
				draw=green!20,
				fill=green!20,
				outer sep=1mm
			] (b) at (9, 0) {};

			\node[
				rectangle,
				text width=170,
				text height=50,
				draw=green!20,
				fill=green!20,
				outer sep=1mm
			] (c) at (19, 0) {};

			\node[
				rectangle,
				text width=170,
				text height=50,
				draw=green!20,
				fill=green!20,
				outer sep=1mm
			] (d) at (19, -7) {};

			\node[
				rectangle,
				text width=100,
				text height=15,
				draw=green!20,
				fill=green!20,
				outer sep=1mm
			] (e) at (9, -7) {};

			\node[
				rectangle,
				text width=150,
				text height=15,
				draw=green!20,
				fill=green!20,
				outer sep=1mm
			] (f) at (0, -7) {};

			\node at (-7, 0) {\small 区間};
			\node at (0, 0) {\small 記号};
			\node at (9, 0.7) {\footnotesize 見出しが同じ};
			\node at (9, -0.7) {\scriptsize 動/形容/助動詞};
			\node at (19, 0.7) {\footnotesize 読みの編集};
			\node at (19, -0.7) {\footnotesize 率が < 50\%};

			\node at (19, -6.3) {\footnotesize 内容語で};
			\node at (19, -7.7) {\footnotesize 品詞が同じ};
			\node at (9, -7) {\small MWE};
			\node at (0, -7) {\small 2語以上};

			\node (p) at (0, -4) {\small 表記};
			\node (q) at (9, -4) {\small 文法};
			\node (r) at (19, -4) {\small 表記};
			\node (s) at (19, -11) {\small 語彙};
			\node (t) at (9, -11) {\small 文法};
			\node (u) at (0, -11) {\small 表現};
			\node (v) at (-8, -7) {\small 文法};

			\draw[-{Triangle[width=30pt,length=16pt]}, line width=2pt] (-5, 0) -- (a.west);
			\draw[-{Triangle[width=30pt,length=16pt]}, line width=2pt] (a.east) -- node [above] {\color{red}No} (b.west);
			\draw[-{Triangle[width=30pt,length=16pt]}, line width=2pt] (b.east) -- node [above] {\color{red}No} (c.west);
			\draw[-{Triangle[width=30pt,length=16pt]}, line width=2pt] (c.east) -- node [above] {\color{red}No} (25, 0) -- (25, -7) -- (d.east);
			\draw[-{Triangle[width=30pt,length=16pt]}, line width=2pt] (d.west) -- node [above] {\color{red}No} (e.east);
			\draw[-{Triangle[width=30pt,length=16pt]}, line width=2pt] (e.west) -- node [above] {\color{red}No} (f.east);
			\draw[-{Triangle[width=30pt,length=16pt]}, line width=2pt] (f.west) -- node [above] {\color{red}No} (v.east);
			\draw[-{Triangle[width=30pt,length=16pt]}, line width=2pt] (a.south) -- node [left] {\color{blue}Yes} (p.north);
			\draw[-{Triangle[width=30pt,length=16pt]}, line width=2pt] (b.south) -- node [left] {\color{blue}Yes} (q.north);
			\draw[-{Triangle[width=30pt,length=16pt]}, line width=2pt] (c.south) -- node [left] {\color{blue}Yes} (r.north);
			\draw[-{Triangle[width=30pt,length=16pt]}, line width=2pt] (d.south) -- node [left] {\color{blue}Yes} (s.north);
			\draw[-{Triangle[width=30pt,length=16pt]}, line width=2pt] (e.south) -- node [left] {\color{blue}Yes} (t.north);
			\draw[-{Triangle[width=30pt,length=16pt]}, line width=2pt] (f.south) -- node [left] {\color{blue}Yes} (u.north);


		\end{tikzpicture}
	\end{figure}
}

\newcommand{\mynum}[2]{%
	{\color{#1}\fontsize{50pt}{0pt}\selectfont\dejavu{\char"#2}\hspace*{-0.35em}}
}


\begin{document}
\begin{frame}[t]
	\begin{columns}[t]
		\begin{column}{.33\linewidth}

			\vspace*{0.5em}

			\begin{mybox}
				\fontsize{50pt}{50pt}\selectfont
				日本語誤り訂正の\kern.0em\textbf{(評価の)}\kern.0emための\textcolor{red}{誤り区間}と\\\textcolor{blue}{誤り種類}の自動アノテーションに向けて

				\begin{flushright}
					\small
					\underline{\textbf{古山翔太}}\kern.0em${}^{1,2}$,
					永田亮\kern.0em${}^{3}$,
					高村大也\kern.0em${}^{2}$,
					岡崎直観\kern.0em${}^{1,2}$
					\\
					(${}^{1}$東工大, ${}^{2}$産総研, ${}^{3}$甲南大)
				\end{flushright}

			\raisebox{60pt}[0pt][0pt]{
				\hspace*{-0.75em}
					\begin{tikzpicture}
						\node[
							rectangle,
							text width = 100,
							text height = 40,
							rounded corners,
							draw=gray!80,
							fill=gray!80,
							outer sep=1mm] at (0, 0) {};
						\node at (0, 0) {\large \textcolor{white!20}{\textbf{P4-9}}};
						\node[overlay] at (6.3, -0.7) {\footnotesize \text{東工大 D1}};
					\end{tikzpicture}
				}
			\vspace*{-1em}

			\end{mybox}

			\begin{block}{\mynum{white}{2776}モデル化:誤り訂正の評価}
				\begin{itemize}
					\item 入力
						\begin{itemize}
							\item モデルの入力・モデルの出力・参照文
						\end{itemize}
					\item 計算するもの
						\begin{itemize}
							\item \textbf{誤り区間の対応関係} ←\textcolor{red}{これをやる}
						\end{itemize}
					\item 出力
						\begin{itemize}
							\item モデルの出力と参照文での一致スコア
						\end{itemize}
				\end{itemize}
				\figa
			\end{block}

			\begin{block}{\mynum{white}{2777}誤り種類はなぜ必要?}
				\begin{itemize}
					\item[\mynum{blue}{2776}] 誤り訂正モデルの評価 (NLP分野)
						\begin{itemize}
							\item 誤り種類別の詳細な評価が可能
						\end{itemize}
					\item[\mynum{blue}{2777}] 誤用分析 (言語教育分野)
						\begin{itemize}
							\item 大規模学習者コーパス分析の自動化
						\end{itemize}
				\end{itemize}
				\figb
			\end{block}

			\begin{block}{\mynum{white}{2778}アノテーション基準:誤り区間}
				\begin{itemize}
					\item 一貫性のある基準として,以下に限定
						\begin{itemize}
							\item 単語・複合語・記号
							\item 助詞・助動詞の連接
							\item multi-word expression (MWE)
							\item 文節
						\end{itemize}
				\end{itemize}
			\end{block}

			\begin{block}{\mynum{white}{2779}アノテーション基準:誤り種類}
				\begin{itemize}
					\item 言語学的分類(4種類)
						\begin{itemize}
							\item 表記・語彙・文法・表現・その他
						\end{itemize}
				\end{itemize}
			\end{block}

		\end{column}

		\begin{column}{.33\linewidth}

			\vspace*{-0.5em}

			\begin{block}{\mynum{white}{277A}誤り区間の推定}
				\begin{itemize}
					\item ERRANT\kern.0emの方法をもとに手法
						\begin{itemize}
							\item 品詞等を考慮した編集距離と動的計画法で単語間の編集を得て,編集をマージして誤り区間を得る
						\end{itemize}
					\item 編集距離の設計\hspace{1em}{\footnotesize(単語分割はspaCy/GiNZA)}\\
						\hspace*{-1em}
						$
						{\text{距離}=\left\{
						\begin{array}{ll}
							0 & \text{  (同じトークン)} \\
							0.5 & \text{  (削除 or 挿入)} \\
							\multicolumn{2}{l}{
									\begin{tabular}{l}
									$0.333 \times (\text{読みの編集率}$ \\
									$                     + \text{見出し} + \text{品詞} )$
									\end{tabular}
								}
						\end{array}
						\right.
						}
						$
						\begin{itemize}
							\item 「読み」が英語との差分
						\end{itemize}
					\item マージ規則
				\end{itemize}
				\vspace*{-0.75em}
				\figc
			\end{block}

			\begin{block}{\mynum{white}{277B}誤り種類の推定}
				\begin{itemize}
					\item 誤り種類の推定
				\end{itemize}
				\vspace*{-1.5em}
				\figd
			\end{block}

			\begin{block}{\mynum{white}{277C}データ作成・評価}
				\begin{itemize}
					\item TEC-JLコーパス中の 100文に対して,アノテーションし,評価
						\begin{itemize}
							\item 改善の余地が大きく,課題が残る
						\end{itemize}
				\end{itemize}


				\begin{table}[H]
				\scalebox{0.95}{
					\phantom{\small 区間+}
					\begin{tabular}{crrrrrr}
						\hline &
						\multicolumn{1}{c}{TP} &
						\multicolumn{1}{c}{FP} &
						\multicolumn{1}{c}{FN} &
						\multicolumn{1}{c}{Prec} &
						\multicolumn{1}{c}{Rec} &
						\multicolumn{1}{c}{F${}_1$} \\
						\hline
						{\small 区間} &
						174 & 108 & 71 &
						61.70 & 71.02 & 66.03 \\
						\hline
					\end{tabular}
					}
				\end{table}

				\begin{table}[H]
				\scalebox{0.95}{
					\begin{tabular}{crrrrrr}
						\hline
						{\small 区間+種類} &
						\multicolumn{1}{c}{TP} &
						\multicolumn{1}{c}{FP} &
						\multicolumn{1}{c}{FN} &
						\multicolumn{1}{c}{Prec} &
						\multicolumn{1}{c}{Rec} &
						\multicolumn{1}{c}{F${}_1$} \\
						\hline {\small  表記 } &  21 &   9 &  27 &  70.00 &  43.75 &  53.85 \\
						\hline {\small  語彙 } &   6 &   5 &   8 &  54.55 &  42.86 &  48.00 \\
						\hline {\small  文法 } & 125 & 116 &  35 &  51.87 &  78.12 &  62.34 \\
						\hline {\small  表現 } &   0 &   0 &  16 &      - &      0 &      0 \\
						\hline {\small その他} &   0 &   0 &   7 &      - &      0 &      0 \\
						\hline {\small   計  } & 152 & 130 &  93 &  53.90 &  62.04 &  57.69 \\
						\hline
					\end{tabular}
				}
				\end{table}


			\end{block}

		\end{column}

		\begin{column}{.33\linewidth}

			\begin{block}{\mynum{white}{277D}課題:誤り区間推定の今後}
				\begin{itemize}
					\item[\mynum{blue}{2776}] 誤り文の単語分割
						\begin{itemize}
							\item 「きれく」は辞書にない
							\item 修正文に依存することがある
								\begin{itemize}
									\item 「かわい\kern.0em|\kern.0emそ\kern.0em|\kern.0emだ」「かわいい\kern.0em|\kern.0emそう\kern.0em|\kern.0emだ」
									\item 「かわいそ\kern.0em|\kern.0emだ」「かわいそう\kern.0em|\kern.0emだ」
								\end{itemize}
						\end{itemize}
					\item[\mynum{blue}{2777}] トップダウンの区間推定へ
						\begin{itemize}
							\item ボトムアップ型 (ERRANT)
								\begin{itemize}
									\item 単語単位の編集をマージ
									\item 誤り区間の一貫性を保ちにくい
								\end{itemize}
							\item トップダウン型
								\begin{itemize}
									\item 区間の候補を先に得て,選択
								\end{itemize}
						\end{itemize}
				\end{itemize}
			\end{block}

			\begin{block}{\mynum{white}{277E}課題:誤り種類推定の今後}
				\begin{itemize}
					\item[\mynum{blue}{2776}] タグの設計
						\begin{itemize}
							\item 下位の分類
							\item 文法誤り
								\begin{itemize}
									\item 品詞 (ERRANT同様)
								\end{itemize}
							\item 表現誤り
								\begin{itemize}
									\item 言い換え
									\item モダリティ(話者の気持ち)の選択
									\item 語用論的な誤り
									\item 論理的な誤り
								\end{itemize}
						\end{itemize}
					\item[\mynum{blue}{2777}] 格関係とモダリティ
						\begin{itemize}
							\item 命題とモダリティの対比に着目{\footnotesize(寺村82)}\hspace*{-2em}
								\begin{itemize}
									\item 「本を読む可能性がある」
									\item 「本を読むかもしれない」
									\item 「本を読」までが「命題」で「格関係」
									\item 後ろが「モダリティ」
								\end{itemize}
							\item 格関係
								\begin{itemize}
									\item 語彙誤り・文法誤り
								\end{itemize}
							\item モダリティ
								\begin{itemize}
									\item 文法誤り・表現誤り
								\end{itemize}
							\item 英語にはない発想が必要
						\end{itemize}
				\end{itemize}
			\end{block}


		\end{column}
	\end{columns}
\end{frame}

\begin{frame}
	\begin{columns}
		\begin{column}{.49\linewidth}

			\begin{mybox}
				\large 補足ポスター
			\end{mybox}

			\begin{block}{誤り種類の仕様はどう決めるか?}
				\begin{itemize}
					\item James 98 の誤り分類法
						\begin{itemize}
							\item[\mynum{orange}{2776}] 逸脱様態に基づく分類 (modification)
								\begin{itemize}
									\item 脱落,付加,誤形成,語順
								\end{itemize}
							\item[\mynum{orange}{2777}] 言語学的分類 (linguistic)
								\begin{itemize}
									\item 表記,文法,語彙,文章,談話
									\item 品詞・文法カテゴリ・同義語対義語・コロケーション等は,その下位分類になる
								\end{itemize}
						\end{itemize}
					\item Granger 03 の誤り種類の設計方針
						\begin{itemize}
							\item[\mynum{orange}{2776}] 有益性・情報提供性 (informativity)
								\begin{itemize}
									\item 学習者のために詳細に情報を提供できる
								\end{itemize}
							\item[\mynum{orange}{2777}] 運用性・管理容易性 (manageablility)
								\begin{itemize}
									\item アノテーターが運用できるよう詳細すぎない
								\end{itemize}
							\item[\mynum{orange}{2778}] 再利性・再利用可能性 (reusablility)
								\begin{itemize}
									\item 異なるドメインでも一般的に利用できる
								\end{itemize}
							\item[\mynum{orange}{2779}] 柔軟性 (flexibility)
								\begin{itemize}
									\item 誤り種類の変更(追加・削除)が容易
								\end{itemize}
							\item[\mynum{orange}{277A}] 一貫性 (consistency)
								\begin{itemize}
									\item アノテーター間での差異がなるべく小さい
								\end{itemize}
						\end{itemize}
				\end{itemize}
			\end{block}

			\begin{block}{参考文献}
				\small
				C. Bryant, M. Felice, T. Briscoe (2017).
				``Automatic Annotation and Evaluation of Error Types for Grammatical Error Correction.''
				% ACL 2017.%, 793-805.
				\\
				C. James (1998).
 				``Errors in Language Learning and Use: Exploring Error Analysis.''
				London and New York: Longman.
				\\
				S. Granger (2003).
				``Error-tagged Learner Corpora and CALL:A Promising Synergy.''
				CALICO Journal 20/3, 465–480.
				\\
				吉川武時 (1997). 
				外国人の日本語誤用分析 誤用分析I.
				明治書院企画編集部\ 編『日本語誤用分析』明治書院.
				\\
				A. Koyama, T. Kiyuna, K. Kobayashi, M. Arai, M. Komachi (2020).
				``Construction of an Evaluation Corpus for Grammatical Error Correction for Learners of Japanese as a Second Language.''
				LREC 2020.
				\\
				宮田学\ 編 (2002).
				ここまで通じる日本人英語 新しいライティングのすすめ.
				大修館書店.
				\\
				A. Lüdeling, M. Walter, E. Kroymann, P. Adolphs  (2005).
				``Multi-level error annotation in learner corpora.''
			\end{block}
		\end{column}

		\begin{column}{.49\linewidth}

			\begin{block}{誤り種類はどう決めたか?}
				\begin{itemize}
					\item 吉川 97 の4分類(言語学的分類)+その他
						\begin{itemize}
							\item 表記・語彙・文法・表現・その他
						\end{itemize}
					\item 運用性・再利性・柔軟性・一貫性を満たす
				\end{itemize}
			\end{block}

			\begin{block}{今後の課題として考えていること}
				\begin{itemize}
					\item[\mynum{blue}{2776}]
						誤り文の単語分割の改善
						\begin{itemize}
							\item 誤り文の単語分割が失敗すると,誤り区間・誤り種類の推定も失敗する
								\begin{itemize}
									\item 「きれく」は辞書にない
								\end{itemize}
							\item 修正文の分割を参照する誤り文の分割は?
								\begin{itemize}
									\item 誤り文の分割は,訂正文の影響を受ける
									\item 「かわい|そ|だ」 $\to$ 「かわいい|そう|だ」
									\item 「かわいそ|だ」 $\to$ 「かわいそう|だ」
								\end{itemize}
						\end{itemize}
					\item[\mynum{blue}{2777}]
						ERRANT\kern.0emの枠組みでの誤り区間の推定の限界は?\hspace*{-1em}
						\begin{itemize}
							\item 単語単位の編集をマージするボトムアップの枠組みは,誤り区間に一貫性を持たせにくい
							\item 基準に適した誤り区間の候補を先に得て,選択するトップダウンの枠組みに可能性?
						\end{itemize}
					\item[\mynum{blue}{2778}]
						評価手法の評価
						\begin{itemize}
							\item より一貫性のあるアノテーション基準の作成
						\end{itemize}
					\item[\mynum{blue}{2779}]
						有益性の高い誤り種類の設計
						\begin{itemize}
							\item 品詞・文法カテゴリ等$\to$誤用分析研究を参考
						\end{itemize}
					\item[\mynum{blue}{277A}]
						「表現の誤り」や「流暢性」などの都合の良い概念に依存していないか?
						\begin{itemize}
							\item 実質的に「その他」と差がなく,有益性が低い
							\item 「論理的誤り」「語用論的誤り」(宮田+ 02)
							\item アノテーションスキームの見直し
								\begin{itemize}
									\item Multi-layer standoff annotation (Lüdeling+ 05)
									\item 日本語に対して,高い表現力のアノテーションに利点があるか?\ (語順・談話などは?)
								\end{itemize}
						\end{itemize}
					\item[\mynum{blue}{277B}]
						係り受け構造の利用
						\begin{itemize}
							\item 構造の違いは,表現の違い
							\item 表現の誤りは,構造が手がかりになる
						\end{itemize}
				\end{itemize}

			\end{block}

		\end{column}

	\end{columns}
\end{frame}

\end{document}

